\section{extended problem}
Let's consider problem 1, 2 mentioned in the introduction.
We shall solve Problem 1 first, then similarly solve Problem 2.


\subsection*{Optimal testing solution for finding a fake coin}
Similar to Section 2, we could present a solution as a graph.

If after testing a graph, we could solve problem 1.
This graph is a testing solution for fake ($TS^-$), feasible. Otherwise it's infeasible.

\begin{definition}
Optimal testing solution for fake ($OTS^-$) is $TS^-$ with minimum number of comparison.

deonte the number of edges of $OTS^-$ as $\tau^-(a,b)$.
\end{definition}

Note that $OTS(a,b)$ could find a real and a fake coin. Thus $\tau(a,b)$
is an upper bound of $\tau^+(a,b)$ and $\tau^-(a,b)$.\\

A $TS^-(a,b)$ better than $OTS(a,b)$ must use another strategy, which is, though all comparisons result in balance we could still find a '-'.\\

Before solving $TS^-$, we observe how a solution works.

When claimming a solution, after these comparisons, there are two cases

\textbf{Case 1.} there is an unbalanced

\textbf{Case 2.} they are all balanced

If \textbf{Case 1.} happens then the solution works.
If \textbf{Case 2.} happens, we must claim a fake coin, and this coin couldn't be a real one in any condition.\\

We can still present a solution as a integer partition.

Example for \textbf{Case 2.} let (a,b)=(2,6)
the optimal solution is [3,3,'1',1] the '1' is the fake one

Denote the part of the fake coin as $i(i\leq\pa)$. 
Beyound the remainning $\pa+\pb-i$ parts, 

(1)they must avoid $\pa$, since if there's a subset sum $\pa$, assign the subset '+', this claim is wrong

(2)they must contain $\pa-i$ or equivalently contain $\pb$, since there must exist a condition such that the coin is fake.


\begin{definition}
For convience, we defone the number of edges of a partition of $n$ as number of edges of the graph transformed in Section 3.
$Q(n,a,b)$ as the number of edges of optimal partition avoids $a$, but contains $b$ 
\end{definition}

Note that $Q(n,a,b)\geq\tau(a,n-a)$\\

There are only \textbf{Case 1.} and \textbf{Case 2.}, 
and for \textbf{Case 2.}, i=1~a for $TS^-$, thus we get the following lemma.

\begin{lemma}
$\tau^-(\pa,\pb)=min\{\tau(\pa,\pb), \{i-1+Q(\pa+\pb-i,\pa,\pb) | 1\leq i\leq\pa\} \}$
\end{lemma}

We are ready to solve $OTS^-$ now.

\begin{theorem}
$OTS^-(\pa,\pb)=optimal\{OTS(a,b), \{\text{a tree of i-1 nodes and }OTS(a,b-i) |1\leq i\leq a\}\}$
\end{theorem}

\begin{proof}
We divide the proof into feasible part and optimal part.

\noindent{\bf (Feasible Part:)}
% $OTS(a,b)$ and $\{\text{a tree of i-1 nodes and }OTS(a,b-i) |for all 1\leq i\leq a\}$ are $TS^-$. 

$OTS(a,b)$ could find a fake coin by definition.

For $\{\text{a tree of i nodes and }OTS(a,b-i) |1\leq i\leq a\}\}$, the i part is real only if 

1. remainning parts contain b-i or 

2. there is at least one unbalanced pair. 

The first condition contradicts definition of $\tau(\pa,\pb-i)$.

If the second condition happens then we get a fake coins from the unbalanced pair, the graph is feasible.

\noindent{\bf (Optimal Part:)}
from lemma 5.2
\begin{align*}
tau^-(a,b)&=min\{tau(a,b),\{i-1+Q(a+b-i,a,b) | 1\leq i\leq a\} \}\\
&\geq min\{tau(a,b), \{i-1+\tau(a,b-i) | 1\leq i\leq a\} \}
\end{align*}

is a optimal lower bound.

And $optimal\{OTS(a,b), \{\text{a tree of i nodes and }OTS(a,b-i) |1\leq i\leq a\}\}$ reaches this lower bound.

\end{proof}

\subsection*{Optimal testing solution for finding a real coin}

$OTS^+$ is trivial.
That is, comparing $\pa$ coins in a group. 

\begin{theorem}
$OTS^+(\pa,\pb)=\text{a tree of }\pa+1\text{ nodes}$
\end{theorem}
\begin{proof}
We divide the proof into feasible part and optimal part.

\noindent{\bf (Feasible Part:)}

If it results in an unbalanced pair. The heavier coin of the pair is the real one.

If it results in all balanced. This means this group is real.

\noindent{\bf (Optimal Part:)}
\begin{align*}
\tau^+(a,b)&=min\{\tau(a,b), \{i-1+Q(a+b-i,b,a) | 1\leq i\leq b\} \}&\\
&\geq min\{\tau(a,b), \{i-1+\tau(a-i,b) | 1\leq i\leq b\} \} &\text{by Theorem 2.2.}\\
&\geq min\{\floor{\frac{a+b}{2}}, i-1+\frac{a-i+b}{2}\}&\\
&\geq min\{a, a\}=a &
\end{align*}

is a optimal lower bound.

$\text{a tree of }\pa+1\text{ nodes}$ reaches optimal lower bound.
\end{proof}