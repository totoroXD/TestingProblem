\section{Integer Partition}

\noindent
To find $g(\pa,\pb)$ with minimal number of edges, it's obvious that $g(\pa ,\pb)$ should not form a cycle

\begin{proof} 
if $g(\pa,\pb)$ form a cycle, remove an edge, the number of vertices each component remains same, whether $g'(\pa,\pb)$ is the subgraph of ${\cal G}(\pa,\pb)$ is equivalent to whether $g(\pa,\pb)$ is the subgraph of ${\cal G}(\pa,\pb)$
\end{proof}

\begin{definition}
We define a multiset avoids $x$ iff its subset-sums has no $x$, a partition avoids $x$ iff its parts avoids $x$
\end{definition}

\begin{theorem}

the problem would be transformed to find a partition of $\pa+\pb$ with  maximal number of parts that avoids $\pa$

\end{theorem}

\begin{proof} 
Let $t =$ number of edge of $g(\pa,\pb)$

$n$ as number of componets in $g(\pa,\pb)$

$M$ as the set of components

$m$ as the set of number of vertices

$m_i$  as number of vertices of $i$th component

since $g(a,b)$ doesn't contain any cycle, $t = \pa+\pb - n$, to find minimal t is equivalent to find maximal n such that $g(\pa,\pb)$ is not the subgraph of ${\cal G}(\pa,\pb)$

if $g(\pa,\pb)$ is the subgraph of ${\cal G}(\pa,\pb)$, 
since each component is conected, each one is either in $K_a$ or $K_b$, there exist a subset of M in $K_a$ the number of vertices of M = a
,e.g there exist a subset of m is equal to $a$

if $g(\pa,\pb)$ is not the subgraph ${\cal G}(\pa,\pb)$, there must not a subset of m is equal to $a$, since if there subset of m is equal to $a$ let these componets in $K_a$ and others in $K_b$, then $g(\pa,\pb)$ is the subgraph ${\cal G}(\pa,\pb)$ thus raise contradiction


the sum of $m$ is $\pa+\pb$, so the theorem holds

\end{proof}

\begin{definition}
$\rho(\pa,\pb)$ is the maximal number of parts of $\pa+\pb$ avoids $\pa$

$d$ is the smallest non-factor of $\pa$
\end{definition}

\begin{remark}
$\tau(\pa,\pb)=\pa+\pb-\rho(\pa,\pb)$
\end{remark}

\begin{theorem}
Optimal partition for $\pa=2$

\[
 \rho(2,\pb) =
   \begin{cases}
     1+\frac{\pb}{3}   &\mbox{if }\pb\equiv0\\
     1+\frac{\pb-1}{3} &\mbox{if }\pb\equiv1\\
     1+\frac{\pb+1}{3} &\mbox{if }\pb\equiv2
   \end{cases}
   \pmod{3}
\]
\end{theorem}

\begin{proof}
	since there is no $a$ in subset sums of the partition $P$, there is at most one '$1$' and no '$2$' in $P$, $|P|\leq \floor{\frac{\pb+2}{3}}$

	for $\pb\equiv0\pmod{3}$, there exist an optimal partition $P=\{1,4,3,...\}$('...' are '$3$'s) that reaches optimal bound and avoids $\pa$

	for $\pb\equiv1\pmod{3}$, there exist an optimal partition $P=\{1,5,3,...\}$('...' are '$3$'s) that reaches optimal bound and avoids $\pa$

	for $\pb\equiv2\pmod{3}$, there exist an optimal partition $P=\{1,3,3,...\}$('...' are '$3$'s) that reaches optimal bound and avoids $\pa$

\end{proof}


\subsection{Exact Algorithm}

\begin{algorithmic}
  \STATE{$R= \{\pa+\pb\}$}
  \FOR{\textbf{each} partition $P$ of the number $\pa+\pb$}
    \IF{$P$ avoids $\pa$ \AND{$|P|>|R|$}} \STATE{$R=P$}
    \ENDIF
  \ENDFOR
  \STATE{output $R$}
\end{algorithmic}
check P's avoidence by Dynamic-Programming as solving discrete knapsack problem, keep avoidence vector when enumerating.

Time Complexity: $O(\pa\frac{exp(\sqrt{\frac{2(\pa+\pb))}{3} } )}{\pa+\pb})$

\pagebreak
\subsection{Results}
\begin{tabular}{l|rrrrrrrrrrrrrrrrrrrr}
 $\rho(\pa,\pb)$ &  1 & 2 &  3 & 4 &  5 & 6 &  7 &  8 &  9 & 10 & 11 & 12 & 13 & 14 & 15 & 16 & 17 & 18 & 19 & 20 \\
\hline
 1        &  1 & 1 &  2 & 2 &  3 & 3 &  4 &  4 &  5 &  5 &  6 &  6 &  7 &  7 &  8 &  8 &  9 &  9 & 10 & 10 \\
 2        &  1 & 2 &  2 & 2 &  3 & 3 &  3 &  4 &  4 &  4 &  5 &  5 &  5 &  6 &  6 &  6 &  7 &  7 &  7 &  8 \\
 3        &  2 & 2 &  3 & 3 &  4 & 3 &  5 &  4 &  6 &  5 &  7 &  6 &  8 &  7 &  9 &  8 & 10 &  9 & 11 & 10 \\
 4        &  2 & 2 &  3 & 4 &  4 & 4 &  4 &  4 &  5 &  5 &  5 &  5 &  6 &  6 &  6 &  7 &  7 &  7 &  8 &  8 \\
 5        &  3 & 3 &  4 & 4 &  5 & 5 &  6 &  5 &  7 &  5 &  8 &  6 &  9 &  7 & 10 &  8 & 11 &  9 & 12 & 10 \\
 6        &  3 & 3 &  3 & 4 &  5 & 6 &  6 &  6 &  6 &  6 &  6 &  6 &  7 &  7 &  7 &  7 &  7 &  7 &  7 &  8 \\
 7        &  4 & 3 &  5 & 4 &  6 & 6 &  7 &  7 &  8 &  7 &  9 &  7 & 10 &  7 & 11 &  8 & 12 &  9 & 13 & 10 \\
 8        &  4 & 4 &  4 & 4 &  5 & 6 &  7 &  8 &  8 &  8 &  8 &  8 &  8 &  8 &  8 &  8 &  9 &  9 &  9 & 10 \\
 9        &  5 & 4 &  6 & 5 &  7 & 6 &  8 &  8 &  9 &  9 & 10 &  9 & 11 &  9 & 12 &  9 & 13 &  9 & 14 & 10 \\
 10       &  5 & 4 &  5 & 5 &  5 & 6 &  7 &  8 &  9 & 10 & 10 & 10 & 10 & 10 & 10 & 10 & 10 & 10 & 10 & 10 \\
 11       &  6 & 5 &  7 & 5 &  8 & 6 &  9 &  8 & 10 & 10 & 11 & 11 & 12 & 11 & 13 & 11 & 14 & 11 & 15 & 11 \\
 12       &  6 & 5 &  6 & 5 &  6 & 6 &  7 &  8 &  9 & 10 & 11 & 12 & 12 & 12 & 12 & 12 & 12 & 12 & 12 & 12 \\
 13       &  7 & 5 &  8 & 6 &  9 & 7 & 10 &  8 & 11 & 10 & 12 & 12 & 13 & 13 & 14 & 13 & 15 & 13 & 16 & 13 \\
 14       &  7 & 6 &  7 & 6 &  7 & 7 &  7 &  8 &  9 & 10 & 11 & 12 & 13 & 14 & 14 & 14 & 14 & 14 & 14 & 14 \\
 15       &  8 & 6 &  9 & 6 & 10 & 7 & 11 &  8 & 12 & 10 & 13 & 12 & 14 & 14 & 15 & 15 & 16 & 15 & 17 & 15 \\
 16       &  8 & 6 &  8 & 7 &  8 & 7 &  8 &  8 &  9 & 10 & 11 & 12 & 13 & 14 & 15 & 16 & 16 & 16 & 16 & 16 \\
 17       &  9 & 7 & 10 & 7 & 11 & 7 & 12 &  9 & 13 & 10 & 14 & 12 & 15 & 14 & 16 & 16 & 17 & 17 & 18 & 17 \\
 18       &  9 & 7 &  9 & 7 &  9 & 7 &  9 &  9 &  9 & 10 & 11 & 12 & 13 & 14 & 15 & 16 & 17 & 18 & 18 & 18 \\
 19       & 10 & 7 & 11 & 8 & 12 & 7 & 13 &  9 & 14 & 10 & 15 & 12 & 16 & 14 & 17 & 16 & 18 & 18 & 19 & 19 \\
 20       & 10 & 8 & 10 & 8 & 10 & 8 & 10 & 10 & 10 & 10 & 11 & 12 & 13 & 14 & 15 & 16 & 17 & 18 & 19 & 20 \\
\hline
\end{tabular}
\begin{tabular}{l|rrrrrrrrrrrrrrrrrrrr}
 $\tau(\pa,\pb)$ &  1 &  2 &  3 &  4 &  5 &  6 &  7 &  8 &  9 & 10 & 11 & 12 & 13 & 14 & 15 & 16 & 17 & 18 & 19 & 20 \\
\hline
 1        &  1 &  2 &  2 &  3 &  3 &  4 &  4 &  5 &  5 &  6 &  6 &  7 &  7 &  8 &  8 &  9 &  9 & 10 & 10 & 11 \\
 2        &  2 &  2 &  3 &  4 &  4 &  5 &  6 &  6 &  7 &  8 &  8 &  9 & 10 & 10 & 11 & 12 & 12 & 13 & 14 & 14 \\
 3        &  2 &  3 &  3 &  4 &  4 &  6 &  5 &  7 &  6 &  8 &  7 &  9 &  8 & 10 &  9 & 11 & 10 & 12 & 11 & 13 \\
 4        &  3 &  4 &  4 &  4 &  5 &  6 &  7 &  8 &  8 &  9 & 10 & 11 & 11 & 12 & 13 & 13 & 14 & 15 & 15 & 16 \\
 5        &  3 &  4 &  4 &  5 &  5 &  6 &  6 &  8 &  7 & 10 &  8 & 11 &  9 & 12 & 10 & 13 & 11 & 14 & 12 & 15 \\
 6        &  4 &  5 &  6 &  6 &  6 &  6 &  7 &  8 &  9 & 10 & 11 & 12 & 12 & 13 & 14 & 15 & 16 & 17 & 18 & 18 \\
 7        &  4 &  6 &  5 &  7 &  6 &  7 &  7 &  8 &  8 & 10 &  9 & 12 & 10 & 14 & 11 & 15 & 12 & 16 & 13 & 17 \\
 8        &  5 &  6 &  7 &  8 &  8 &  8 &  8 &  8 &  9 & 10 & 11 & 12 & 13 & 14 & 15 & 16 & 16 & 17 & 18 & 18 \\
 9        &  5 &  7 &  6 &  8 &  7 &  9 &  8 &  9 &  9 & 10 & 10 & 12 & 11 & 14 & 12 & 16 & 13 & 18 & 14 & 19 \\
 10       &  6 &  8 &  8 &  9 & 10 & 10 & 10 & 10 & 10 & 10 & 11 & 12 & 13 & 14 & 15 & 16 & 17 & 18 & 19 & 20 \\
 11       &  6 &  8 &  7 & 10 &  8 & 11 &  9 & 11 & 10 & 11 & 11 & 12 & 12 & 14 & 13 & 16 & 14 & 18 & 15 & 20 \\
 12       &  7 &  9 &  9 & 11 & 11 & 12 & 12 & 12 & 12 & 12 & 12 & 12 & 13 & 14 & 15 & 16 & 17 & 18 & 19 & 20 \\
 13       &  7 & 10 &  8 & 11 &  9 & 12 & 10 & 13 & 11 & 13 & 12 & 13 & 13 & 14 & 14 & 16 & 15 & 18 & 16 & 20 \\
 14       &  8 & 10 & 10 & 12 & 12 & 13 & 14 & 14 & 14 & 14 & 14 & 14 & 14 & 14 & 15 & 16 & 17 & 18 & 19 & 20 \\
 15       &  8 & 11 &  9 & 13 & 10 & 14 & 11 & 15 & 12 & 15 & 13 & 15 & 14 & 15 & 15 & 16 & 16 & 18 & 17 & 20 \\
 16       &  9 & 12 & 11 & 13 & 13 & 15 & 15 & 16 & 16 & 16 & 16 & 16 & 16 & 16 & 16 & 16 & 17 & 18 & 19 & 20 \\
 17       &  9 & 12 & 10 & 14 & 11 & 16 & 12 & 16 & 13 & 17 & 14 & 17 & 15 & 17 & 16 & 17 & 17 & 18 & 18 & 20 \\
 18       & 10 & 13 & 12 & 15 & 14 & 17 & 16 & 17 & 18 & 18 & 18 & 18 & 18 & 18 & 18 & 18 & 18 & 18 & 19 & 20 \\
 19       & 10 & 14 & 11 & 15 & 12 & 18 & 13 & 18 & 14 & 19 & 15 & 19 & 16 & 19 & 17 & 19 & 18 & 19 & 19 & 20 \\
 20       & 11 & 14 & 13 & 16 & 15 & 18 & 17 & 18 & 19 & 20 & 20 & 20 & 20 & 20 & 20 & 20 & 20 & 20 & 20 & 20 \\
\hline
\end{tabular}

\pagebreak