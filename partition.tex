\documentclass[12pt,a4paper]{article}
\usepackage{fontenc}[T1]
\usepackage{algorithm}
\usepackage{algorithmic}
\usepackage{amsfonts}
\usepackage{amsmath, array}
\usepackage{amsthm}
\usepackage{vmargin}
\usepackage{multicol}
\usepackage{graphicx}
\usepackage{enumerate}
\setmargrb{19.0mm}{12.0mm}{19.0mm}{12.0mm}%{left}{up}{right}{down}

\newtheorem{theorem}{Theorem}
\newtheorem{definition}{Definition}

\title{}
\author{}
\begin{document}


\noindent
To find $g(a,b)$ with minimal number of edges, it's obvious that $g(a,b)$ should not form a cycle

\begin{proof} 
if $g(a,b)$ form a cycle, remove an edge, the number of vertices each component remains same, whether $g'(a,b)$ is the subgraph of ${\cal G}(a,b)$ is equivalent to whether $g(a,b)$ is the subgraph of ${\cal G}(a,b)$
\end{proof}

\begin{theorem}

the problem would be transformed to find the maximal number of partitions of $a+b$ such that no subset sum of these partition equals $a$

\end{theorem}

\begin{proof} 
Let $t =$ number of edge of $g(a,b)$

$n$ as number of componets in $g(a,b)$

$M$ as the set of components

$m$ as the set of number of vertices

$m_i$  as number of vertices of $i$th component

since $g(a,b)$ doesn't contain any cycle, $t = a+b - n$, to find minimal t is equivalent to find maximal n such that $g(a,b)$ is not the subgraph of ${\cal G}(a,b)$

if $g(a,b)$ is the subgraph of ${\cal G}(a,b)$, 
since each component is conected, each one is either in $K_a$ or $K_b$, there exist a subset of M in $K_a$ the number of vertices of M = a
,e.g there exist a subset of m is equal to $a$

if $g(a,b)$ is not the subgraph ${\cal G}(a,b)$, there must not a subset of m is equal to $a$, since if there subset of m is equal to $a$ let these componets in $K_a$ and others in $K_b$, then $g(a,b)$ is the subgraph ${\cal G}(a,b)$ thus raise contradiction


the sum of $m$ is $a+b$, so the theorem holds

\end{proof}

\end{document}