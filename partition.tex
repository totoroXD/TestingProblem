%!TEX root = total.tex

\section{From Graph to Integer Partition}
This section shows an interesting connection between designing an optimal foolproof scheme and 
searching for an integer partitioning of some special form.  
We begin with the following simple lemma, which is crucial in establishing the connection.

\begin{lemma} \label{lem:no-cycle}
If  ${\cal S}$ is an optimal foolproof scheme, then its corresponding graph $G_{\cal S}$ does 
not contain any cycle.  
\end{lemma}
\begin{proof}
Suppose on the contrary that $G_{\cal S}$ does.  Then,  we can obtain a graph $G'$ by removing an edge from some cycle in $G_{\cal S}$.  If $G'$ is not be isomorphic to any subgraph of ${\cal G}(\pa,\pb)$, by Theorem~\ref{thm:graph}, this would imply $G'$ corresponds to a foolproof scheme;  furthermore, such a scheme performs fewer comparisons than ${\cal S}$, 
so that ${\cal S}$ is not optimal.   Otherwise, if $G'$ is isomorphic to some subgraph of ${\cal G}(\pa,\pb)$, then $G_{\cal S}$ would also be isomorphic to some subgraph of ${\cal G}(\pa,\pb)$ (using the same mapping between the vertices), 
so that ${\cal S}$ is not foolproof by Theorem~\ref{thm:graph}.  
Contradiction occurs in both cases, and the lemma thus follows. 
\end{proof}

Based on the above lemma, the graph $G_{\cal S}$ corresponding to an optimal foolproof scheme ${\cal S}$ must be a \emph{forest}.  Indeed, we may observe that the \emph{shape} of each connected tree is not important:  
Precisely, for each tree $T$ in $G_{\cal S}$ that connects some set $U$ of vertices, we can replace $T$ 
by any other tree that connects the vertices in $U$, and the resulting scheme remains foolproof.\footnote{%  
If not, the latter graph is isomorphic to some subgraph of ${\cal G}(\pa,\pb)$, but then $G_{\cal S}$ would also be
isomorphic to some subgraph of ${\cal G}(\pa,\pb)$ under the same vertex mapping.}  
Also, after replacing $T$, the new scheme remains optimal as it uses the same number of comparisons.
This naturally implies that an optimal foolproof scheme is related to some kind of \emph{partitioning} 
of the integer $\pa + \pb$.  In the following, we shall unveil the property of such a partitioning.  
We first define a related concept.

\begin{definition}
Let $P$ be a multiset of positive integers.  
We say $P$ \emph{avoids} a positive integer $x$ if for any subset $P' \subseteq P$,
the sum of all integers in $P'$ is not equal to $x$.
\end{definition}

Let $P = \{ p_1, p_2, \ldots, p_{|P|} \}$ be a multiset of positive integers whose sum is $\pa + \pb$.  
In other words, $P$ forms an integer partition of $\pa + \pb$.   
We say a testing scheme ${\cal S}$ \emph{corresponds to} $P$ if $G_{\cal S}$ is a forest whose trees have 
sizes $p_1, p_2 \ldots, p_{|P|}$, respectively.  Then, we have the following theorem.
 
\begin{theorem} \label{thm:partition}
Let ${\cal S}$ be a testing scheme whose corresponding graph $G_{\cal S}$ is a forest.
Then ${\cal S}$ is foolproof if and only if ${\cal S}$ corresponds to a partition $P$ of the integer $\pa+\pb$
that avoids $\pa$.
\end{theorem}
\begin{proof}
We prove the necessary and sufficient conditions separately, each by contradiction.

\medskip

\noindent
{\bf ($\Rightarrow$)}
Suppose that ${\cal S}$ is foolproof.  Assume on the contrary that $P$ does not avoid $\pa$, then we can partition $P$ into two subsets $P_\pa$ and $P_\pb$ such that the sum of integers in $P_\pa$ is equal to $\pa$ (and thus, the sum of integers in $P_\pb$ is $\pb$).   
Then, consider those trees in $G_{\cal S}$ with sizes $p \in P_\pa$, they together would be isomorphic to some subgraph of $K_\pa$;  similarly, the remaining trees in $G_{\cal S}$ would be isomorphic to some subgraph of $K_\pb$.  
This implies that $G_{\cal S}$ is isomorphic to some subgraph of ${\cal G}(\pa,\pb)$, so that ${\cal S}$ is not foolproof by Theorem~\ref{thm:graph}.  A contradition occurs.

\medskip

\noindent
{\bf ($\Leftarrow$)} Suppose that $P$ avoids $\pa$.  Assume on the contrary that ${\cal S}$ is not foolproof.  
By Theorem~\ref{thm:graph}, $G_{\cal S}$ is isomorphic to some subgraph of ${\cal G}(\pa,\pb)$.   
Consider a particular isomorphism $f$.  Note that the number of vertices of both $G_{\cal S}$ and ${\cal S}(\pa,\pb)$ are the same, so that $f$ is a bijection between the two vertex sets.  Based on $f$, we can partition the vertices in $G_{\cal S}$ into two groups, where the first group contains those who are mapped to vertices in the $K_\pa$ component of ${\cal G}(\pa,\pb)$, and the second one contains those remaining vertices.  Also, all vertices from the same tree in $G_{\cal S}$ must be in the same group.  By focussing on those trees whose vertices are mapped to the first group, their sizes adds up to $\pa$, so that 
$P$ does not avoid $\pa$.  A contradiction occurs.
\end{proof}

For instance, although the scheme corresponding to Figure~\ref{fig:nonfoolproof} is not foolproof when $\pa = 3$ and $\pb = 5$, it is foolproof when $\pa = 1$ and $\pb = 7$, or when $\pa = 4$ and $\pb = 4$.  Also, we have the following corollary.

\begin{corollary} \label{cor:maximize-|P|}
Let ${\cal S}$ be an optimal testing scheme.
Then, ${\cal S}$ corresponds to a partition $P_{\max}$ of $\pa+\pb$, with the maximal number of parts, that avoids $\pa$; furthermore, $\tau(\pa,\pb) = \pa + \pb - |P_{\max}|$, where the notation $|P|$ denotes the number of parts in a partition $P$.
\end{corollary}
\begin{proof}
By Lemma~\ref{lem:no-cycle}, $G_{\cal S}$ is a forest.  
Then, by Theorem~\ref{thm:partition}, ${\cal S}$ corresponds to a partition $P$ of $\pa + \pb$ that avoids $\pa$.
Furthermore, the number of comparisons performed by ${\cal S}$
is exactly $\pa + \pb - |P|$.  As ${\cal S}$ is optimal, this implies ${\cal S}$ minimizes the number of comparisons, 
so that the corresponding partition $P$ maximizes the number of parts.  The corollary thus follows.
\end{proof}

\begin{corollary}
For any $\pb \geq 1$, 
\[
    \tau(2,\pb) = \pb + 2 - \myfloor{\frac{\pb+2}{3}}.
\]
\end{corollary}
\begin{proof}
For any partition $P$ of $2 + \pb$ that avoids $2$, $P$ contains at most one $1$ and no $2$s, so that 
\[ |P|\  \leq\ \max\, \left\{\, 1 + \frac{\pb+1}{3},\ \; \frac{\pb+2}{3}\ \right\}\ \leq\ 1 + \frac{\pb+1}{3} \ =\ \frac{\pb+4}{3}.\]  
Furthermore, since $|P|$ is an integer, the above inequality implies that
\[ |P|\ \leq\ \myceil{\frac{\pb+4}{3}}\ =\ \myfloor{\frac{\pb+2}{3}}.\footnote{%
The equality follows from the fact: for any integer $n$ and any positive integer $m$, 
$\myceil{n/m} = \myfloor{(n+m-1)/m}$.}
\]
Thus by Corollary~\ref{cor:maximize-|P|}, we have  $\tau(2,\pb) \geq \pb + 2 - \myfloor{(\pb+2)/3}$.

\medskip

\noindent
However, by setting $P$ to be
\begin{itemize}
\setlength{\itemsep}{-1pt}
\item  $P=\{3,3, \ldots \}$ \ \ \ (where $\ldots$ represents trailing $3$s) for $\pb + 2 \equiv 0 \  (\bmod\ 3)$, or
\item  $P=\{4,3, \ldots \}$ \ \ \ (where $\ldots$ represents trailing $3$s) for $\pb + 2 \equiv 1 \  (\bmod\ 3)$, or
\item  $P=\{1,4,3, \ldots \}$ (where $\ldots$ represents trailing $3$s) for $\pb + 2 \equiv 2 \  (\bmod\ 3)$,
\end{itemize}
each partition $P$ is a partition of $2 + \pb$ that avoids $2$, and with $|P| = \myfloor{(\pb+2)/3}$.
By Corollary~\ref{cor:maximize-|P|}, this implies that $\tau(2,\pb) \leq \pb + 2 - \myfloor{(\pb+2)/3}$.  
As $\tau(2,\pb)$ is now upper-bounded and lower-bounded by the same desired quantity, 
the corollary thus follows.
\end{proof}

