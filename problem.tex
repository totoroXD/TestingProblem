\documentclass[12pt,a4paper]{article}
\usepackage{algorithm}
\usepackage{algorithmic}
\usepackage{amsfonts}
\usepackage{vmargin}
\setmargrb{19.0mm}{12.0mm}{19.0mm}{12.0mm}%{left}{up}{right}{down}


\begin{document}


Question:
Now we have a's "+" and b's "-", each time we can choose 2 of them to check if they are the same or not.  $\tau(a,b)$ is the minimum number of testing that we can garantee to find a different pair "+" and "-".  \\


Idea:
Consider a graph G(a,b) contains two part, one is $K_a$, another is $K_b$, then we want to find a graph g(a,b) (vetex is not greater than a+b), ang g(a,b) is not a subgraph of G(a,b). Let $t =$ numer of edge of g(a,b), then through  $t$ testing we can   garantee to find a different pair "+" and "-".  \\

Proof:
We tesing as edge of g(a,b). If we can't find the pair "+" and "-" after "t" testing, we can assign a's "+" and b's "-" into g(a,b),  then g(a,b) is the subgraph og G(a,b). By contradiction. \\

Lemma:
 $\tau(a,b)= min(t)$ \\

Proof:
For any graph contains the nunber of edges is less than $min(t)$, imply the graph is a subgraph of G(a,b), then we can conclude that exist a probability greater than 0 that we can not  find a different pair "+" and "-".  \\

Conclusion:
We  want to find a graph g(a,b) that has minimum edges and g(a,b) is not the subgraph of G(a,b), then number of edges of g(a,b) is the answer. \\ 






\end{document}