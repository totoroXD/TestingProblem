%!TEX root = total.tex

\section{Finding Optimal Foolproof Scheme}
Based on Corollary~\ref{cor:maximize-|P|} in the previous section, we can find an optimal foolproof scheme 
by enumerating all possible partitions of $\pa+\pb$, and selecting among those that avoids $\pa$ the one 
that consists of the most number of parts.  This approach is equivalent to solving the NP-complete 
{\sc Subset Sum} problem~\cite{GarJoh79} for each possible partition of $\pa+\pb$, 
which can be done via dynamic programming~\cite{CLRS09} as shown in Algorithm~\ref{algo:naive}:

\begin{algorithm}
\caption{Find Optimal Foolproof Scheme (OFS)}
\label{algo:naive}
\begin{algorithmic}
  \Procedure{Find\_Optimal\_Foolproof\_Scheme}{$\pa,\pb$}\Comment{Return an OFS}
    \State{Set default partition $R$ as $\{\pa+\pb\}$;}
    \For{\textbf{each} partition $P$ of the number $\pa+\pb$ which avoids $\pa$}
      \If{Subset\_Sum($P,\pa$)} \State \bf continue
      \ElsIf{$|P|>|R|$} \State{update $R$ as $P$;}
      \EndIf
    \EndFor
    \State{\textbf{return} $R$}
  \EndProcedure
  \end{algorithmic}
\begin{algorithmic}
  \Procedure{Subset\_Sum}{$P,\pa$}\Comment{Check if a subset of $P$ sums up to $\pa$}
    \State Set default multiset $S$ as $\{0\}$;
    \For{\textbf{each} integer $\pi \in P$} 
      \State{$S \leftarrow S \cup \{\pi + s \mid s \in S \mbox{\ and\ } \pi + s \leq \pa \}$;}
    \EndFor
    \State \textbf{return} $a \in S$
  \EndProcedure
\end{algorithmic}
\end{algorithm}

Let $p(n)$ denote the number of different partitions of $n$, 
which asymptotically approaches $\smash{\mathrm{exp}(\pi\sqrt{2n/3})/(4n\sqrt{3})}$ 
as $n \rightarrow \infty$~\cite{HarRam18}.   
The running time of the procedure Subset\_Sum$(P,\pa)$
is $|P| \times O(\pa)$, which is bounded by $O((\pa + \pb)\pa)$.  Thus, by Algorithm~\ref{algo:naive}, 
we can find an optimal foolproof scheme, as well as compute the value of $\tau(\pa,\pb)$, in 
$$p(\pa+\pb) \times O((\pa+\pb)\pa) = O\left(\pa \ e^{\pi\sqrt{2(\pa+\pb)/3}}\right) \mbox{\ time.}$$  
\ref{sec:results} shows a table of $\tau(\pa,\pb)$ for $1 \leq \pa, \pb \leq 15$.

\medskip

\noindent
{\it Remark.} If we enumerate the partitions by depth-first search (according to lexicographical order), 
                      consecutive partitions during the enumeration will generally be similar, so that 
                      the time for Subset\_Sum$(P,\pa)$ for each partition requires amortized $O(\pa)$ time.
                     Consequently, this allows us to reduce the time complexity further by a factor of $O(\pa+\pb)$.
                     Nevertheless, this is not the main focus of the work, so that we omit the details for brevity.

\subsection{Cyclic Property of $\tau(\pa,\pb)$}
In this subsection, we shall show that for any $\pa$, 
there is some property that an optimal foolproof scheme must possess when $\pb$ is sufficiently large.
Consequently, this allows us to show a cyclic property about $\tau(\pa,\pb)$, 
as well as speed up the construction time of an optimal foolproof scheme. 

\medskip

\noindent
Before delving into details, let us take a look about the values $\tau(2,\pb)$ and $\tau(5,\pb)$ for different $\pb$s.

\medskip

\begin{tabular}{c|rrrrrrrrrrrrrrr}
$\pb$ &  1 &  2 &  3 &  4 &  5 &  6 &  7 &  8 &  9 & 10 & 11 & 12 & 13 & 14 & 15  \\
\hline
$\tau(2,\pb)$  &  2 &  2 &  3 &  4 &  4 &  5 &  6 &  6 &  7 &  8 &  8 &  9 & 10 & 10 & 11 \\
$\tau(5,\pb)$        &  3 &  4 &  4 &  5 &  5 &  6 &  6 &  8 &  7 & 10 &  8 & 11 &  9 & 12 & 10  \\
\hline
\end{tabular}

\medskip

\noindent
From $\pb\geq 1$, $\tau(2,\pb)$ forms cycle of length $3$, such that $\tau(2,\pb+3) = \tau(2,\pb) + 2$.
Similarly, from $\pb\geq 9$, $\tau(5,\pb)$ forms cycle of length $2$, such that $\tau(5,\pb+2) = \tau(5,\pb) + 1$.
In fact, for any fixed $\pa$, such kind of cycles must eventually occur.  
To see why this is true, we shall now examine more closely about the structure of an optimal foolproof scheme.  

\medskip

\noindent
Let $\ell(n)$ denote the least non-factor of $n$, which is the smallest positive integer that does not divide $n$ properly.
For instance, $\ell(1) = 2$, $\ell(2) = 3$, $\ell(3) = 2$, $\ell(4) = 3$, $\ell(5) = 2$, and $\ell(6) = 4$.  
To simplify our discussion, we shall now assume $\pa \leq \pb$ throughout this subsection, and use $\pl$ to denote $\ell(\pa)$.
Also, we call a partition $P$ of $n$ a \emph{maximal} partition that avoids $x$, if among all partitions of $n$ 
which avoid $x$, $P$ is one that contains the maximum number of parts;  furthermore, we use $\rho(x, n)$ to denote its number of parts.
The following two lemmas give some basic properties about a partition that avoids $\pa$.

\begin{lemma} \label{lem:at-least-b/L}
Let $P$ be a maximal partition of $\pa + \pb$ that avoids $\pa$.  
Then, $|P| = \rho(\pa, \pa+\pb) \geq \myceil{\pb/\pl}$.
\end{lemma} 
\begin{proof}
Consider a partition of $\pa+\pb$ with $\myceil{\pb/\pl}-1$ instances of $\pl$ and a remainder part $r = \pa+\pb-\pl(\myceil{\pb/\pl}-1)$.  
For any subset $P'$ of this partition, either $P'$ contains $r$, or $P'$ contains purely copies of $\pl$.  
In the former case, the subset sum of $P'$ cannot be $\pa$, since $r > \pa$. 
In the latter case, the subset sum of $P'$ cannot be $\pa$, since $\pl$ does not divide $\pa$.
Thus, the above partition avoids $\pa$, and contains $\myceil{\pb/\pl}$ parts.   
This completes the proof of the lemma.
\qed
\end{proof}

\begin{lemma} \label{lem:at-most-a-1}
If a partition $P$ avoids $\pa$, $P$ contains less than $\pa$ integers that is a factor of $\pa$;  
in other words, $| \{ \pi \mid \pi \in P \mbox{\ and\ } \pi < \pl \}|  < \pa$. 
\end{lemma}
\begin{proof}
Let $P = \{ \pi_1, \pi_2, \ldots \}$ with $\pi_i \leq \pi_j$ when $i < j$.  
Moreover, let $k$ be the number of integers in $P$ that is a factor of $\pa$
(i.e., $k$ is the largest index such that $\pi_k < \pl$).   

\medskip

\noindent
Let $S_i$ denote the set of integers in $[1,\pa]$ that can be written as a sum from a subset of
integers in $P_i = \{ \pi_1, \pi_2, \ldots, \pi_i \}$.   Thus, if $\pi_1 < \pl$,
we have $S_1 = \{ \pi_1 \}$, and $|S_1| = 1$.
We claim that $|S_i| < |S_{i+1}|$ for any $i \in [1,k-1]$, so that $|S_k| \geq k$.  
Furthermore, since $P$ avoids $\pa$, we must have $|S_k| < \pa$, so that we obtain
the desired relationship that $k < \pa$.

\medskip

\noindent
It remains to prove the claim.  Suppose on the contrary that $|S_i| = |S_{i+1}|$ for some $i \in [1,k-1]$.  
This implies that $\pi_{i+1} \in S_i$, or else $S_{i+1}$ contains at least one more integer than $S_i$.
As $\pi_{i+1} \in S_i$, we must now have $2\cdot \pi_{i+1} \in S_i$ for the same reason.
Inductively, for each $t \geq 1$ such that $t\cdot \pi_{i+1} \leq \pa$, we must have $t\cdot \pi_{i+1} \in S_i$.
Since $\pi_{i+1}$ is a factor of $\pa$, this implies that $\pa \in S_i$.  In other words, $\pa$
can be written as a sum of a subset of integers in $P_i \subseteq P$; this contradicts with the fact that
$P$ avoids $\pa$.  Thus, the claim (and the lemma) follows.
\qed
\end{proof}

Based on the above lemmas, we are ready to establish 
the key lemma and the main theorem of this subsection as follows.

\begin{lemma}  \label{lem:key-lemma}
Let $P$ be a maximal partition of $\pa + \pb$ that avoids $\pa$.
If $\pb > \pl\left((\pl+1)(\pa - 1) + \myceil{\pa/\pl}\right)$,
$P$ must contain at least $\myceil{\pa/\pl} = \myfloor{\pa/\pl} + 1$ instances of $\pl$.
\end{lemma}
\begin{proof} 
We shall prove this by contradiction.  Let us fix a maximal partition~$P$, and use $X_i$ to denote
the number of instances of $i$ contained in $P$.  
Assume on the contrary that $X_\pl \leq \myfloor{\pa/\pl}$.  Then, we have
\[
  |P| = \sum_{1\leq i < \pl}{X_i} + X_\pl +\sum_{i \geq \pl+1}{X_i}.
\]
Since $\sum_i {i \cdot X_i} = \pa+\pb$, under the conditions that  
$\sum_{1\leq i < \pl}{X_i} \leq \pa-1$ by Lemma~\ref{lem:at-most-a-1} and 
$X_\pl \leq \myfloor{\pa/\pl}$ by the assumption, we get
\begin{eqnarray*}
  |P|\ &\leq& \  
       \pa - 1 + \myfloor{\frac{\pa}{\pl}}   +  \frac{(\pa + \pb) - 1 \cdot (\pa-1) - \pl \cdot (\myfloor{\pa/\pl})}{\pl+1}\\
       &=& \  \pa - 1 + \frac{\pb  + 1 + \myfloor{\pa/\pl}}{\pl+1}\ \ =\ \ \pa - 1 + \frac{\pb + \myceil{\pa/\pl}}{\pl+1}.
\end{eqnarray*}  
By Lemma~\ref{lem:at-least-b/L}, we have $|P| \geq \myceil{\pb/\pl} \geq \pb/\pl$, so that
\[
  \frac{\pb}{\pl}\ \ \leq\ \ \pa - 1 + \frac{\pb + \myceil{\pa/\pl}}{\pl+1}.
\]
Multiplying both sides by $\pl(\pl+1)$, we obtain
\[
  \pb(\pl+1)\ \ \leq\ \ \pl(\pl+1)(\pa - 1) + \pb\pl + \pl\myceil{\pa/\pl}.
\]
This contradicts with the condition that $\pb > \pl \left((\pl+1)(\pa-1) + \myceil{\pa/\pl}\right)$, 
and the lemma thus follows.
\qed
\end{proof}

\begin{theorem} \label{thm:OFS-of-P+L}
Let $P$ be a maximal partition of $\pa + \pb$ that avoids $\pa$.
If $\pb > \pl\left((\pl+1)(\pa - 1) + \myceil{\pa/\pl}\right)$, then
$P' = P \cup \{ \pl \}$ must be a maximal partition of $\pa + \pb + \pl$ that avoids $\pa$.
\end{theorem}
\begin{proof}
We first show that $P'$ avoids $\pa$, and then further show that $P'$ is a maximal partition.
Both facts together will complete the proof.

\medskip

\noindent
{\bf $P'$ avoids $\pa$:}  Suppose on the contrary that some subset of $P'$ adds up to $\pa$.  This happens only if some subset of $P$
adds up to $\pa$ or $\pa - \pl$.  The former case cannot occur since $P$ avoids $\pa$.  For the latter case, let $S$ be a subset of $P$ 
whose sum is exactly $\pa - \pl$.  Then, $S$ contains at most $\myceil{\pa/\pl} - 1$ instances of $\pl$ (otherwise the sum exceeds $\pa - \pl$);  
however, by Lemma~\ref{lem:key-lemma}, $P$ contains at least $\myceil{\pa/\pl}$ copies of $\pl$, so that $\pl \in P\backslash S$.
Consequently, $S \cup \{\pl\} \subseteq P$, whose sum is exactly $\pa$.  This contradicts with the assumption that $P$ avoids $\pa$.

\medskip

\noindent
{\bf $P'$ is maximal:}  Suppose on the contrary that there exists some partition $S'$ of $\pa+\pb+\pl$ that avoids $\pa$, and with $|S'| > |P'|$.
By Lemma~\ref{lem:key-lemma}, $S'$ contains at least one instance of $\pl$.  Removing one instance of $\pl$ from $S'$ would give a 
partition $S$ of $\pa + \pb$ that avoids $\pa$, and with $|S| = |S'| - 1 > |P'| - 1 = |P|$.  This contradicts with the maximality of $P$.
\qed
\end{proof}

The above theorem, together with Corollary~\ref{cor:maximize-|P|}, immediately leads to the following corollary:

\begin{corollary}
For sufficiently large $\pb$,  
\begin{eqnarray*}
  \rho(\pa, \pa + \pb + \pl) &=& \rho(\pa + \pb) + 1 \\
  \tau(\pa,\pb+\pl)             &=& \tau(\pa,\pb) + \pl - 1.
\end{eqnarray*}
\end{corollary}

\subsection{Speed-Up in Finding Optimal Foolproof Scheme}
Based on Theorem~\ref{thm:OFS-of-P+L}, we can find an optimal foolproof scheme in an alternative way, as shown in Algorithm~\ref{algo:faster}.  
Basically, if $\pb$ is sufficiently large, we will first calculate a maximum $k$ such that $\pb - k\pl$ remains sufficiently large, 
and construct a maximal partition $P$ of $\pa + \pb - k\pl$ that avoids $\pa$. 
Then, we extend $P$ to a maximal partition $P'$ of $\pa + \pb$ by adding $k$ copies of $\pl$.  Else, we will simply run Algorithm~\ref{algo:naive}
to obtain the answer.  Consequently, for any value of $\pb$, the running time of the algorithm is bounded by 
$$p(\pa + \pl((\pl+1)(\pa-1) + \myceil{\pa/\pl})) \times O(\pa) \approx O\left(e^{\pi\pl\sqrt{2\pa/3}}\right),$$  
which is independent of $\pb$.

\begin{algorithm}
\caption{A Faster Algorithm to Find Optimal Foolproof Scheme}\label{algo:faster}
\begin{algorithmic}
  \Procedure{Faster\_Optimal\_Foolproof\_Scheme}{$\pa,\pb$}\Comment{Return an OFS}
    \If {$\pb > \pl((\pl+1)(\pa-1) + \myceil{\pa/\pl})$}\Comment{$\pl$ is the least non-factor of $\pa$}
        \State {$k \leftarrow$ maximum integer such that $\pb - k\pl > \pl((\pl+1)(\pa-1) + \myceil{\pa/\pl})$;}
        \State {$R\ \leftarrow\ $ Find\_Optimal\_Foolproof\_Scheme$(\pa,\pb - k\pl)$; }         
        \State {\textbf{return} $R \cup \{ \pl, \ldots, \pl \}$; }\Comment{$\{ \pl, \ldots, \pl \}$ = $k$ copies of $\pl$}
    \Else
         \State {\textbf{return} Find\_Optimal\_Foolproof\_Scheme$(\pa,\pb)$;}
    \EndIf
  \EndProcedure
\end{algorithmic}
\end{algorithm}

