\section{Computing $\tau(\pa,\pb)$}
To find optimal, we enumerate all partitions of $\pa+\pb$.
Then checking P's avoidence, we solve this as below. This idea is same as solving discrete knapsack problem by DP. This checking runs in $O(\pa\pb)$ for a particular partition.

\begin{algorithm}
\caption{Find optimal foolproof scheme}\label{alg:stupid}
\begin{algorithmic}
  \Procedure{Find\_optimal\_foolproof\_scheme}{$\pa,\pb$}\Comment{return a OFS}
    \State{$R= \{\pa+\pb\}$}
    \For{\textbf{each} partition $P$ of the number $\pa+\pb$ which avoids $\pa$}
      \If{Check\_avoidence($P,\pa$)} \State \bf continue \EndIf
      \If{$|P|>|R|$} \State{$R=P$}
      \EndIf
    \EndFor
    \State{\textbf{return} $R$}
  \EndProcedure
  \end{algorithmic}
\begin{algorithmic}
  \Procedure{Check\_avoidence}{$P,\pa$}\Comment{Whether $P$ avoids $\pa$}
    \State $S= \{0\}$
    \For{\textbf{each} number $pi\in P$}\Comment{$O(\pa+\pb) times$}
      \State $S=S\cup S\text{\d{+}}pi$\Comment{This cost $O(\pa)$}
    \EndFor
    \State \textbf{return} $a\not\in S$
  \EndProcedure
\end{algorithmic}
\end{algorithm}

Denote the number of partitions of $n$ as $p(n)$, the number of partitions avoids $a$ of $n$ as $p_{aoivds\ a}(n)$

Time Complexity: 

\[
O(\pa\pb\times p(\pa+\pb))\\
=O(\pa\pb\frac{exp(\sqrt{\frac{2(\pa+\pb))}{3} } )}{\pa+\pb})
\]

We can improve this a little by enumerating only partitions which avoids $\pa$.
\begin{algorithm}
\caption{Find optimal foolproof scheme}\label{alg:naive}
\begin{algorithmic}
  \Procedure{Find\_optimal\_foolproof\_scheme}{$\pa,\pb$}\Comment{return a OFS}
    \State{$R= \{\pa+\pb\}$}
    \For{\textbf{each} partition $P$ of the number $\pa+\pb$ which avoids $\pa$}
      \If{$|P|>|R|$} \State{$R=P$}
      \EndIf
    \EndFor
    \State{output $R$}
  \EndProcedure
  \end{algorithmic}
\end{algorithm}

Time Complexity: 

\[
O(a\times p_{aoivds\ \pa}(\pa+\pb))\\
\leq O(\pa\times p(\pa+\pb))\\
\leq O(\pa\frac{exp(\sqrt{\frac{2(\pa+\pb))}{3} } )}{\pa+\pb})
\]
\begin{remark}
We shall enumerate partitions by Depth-First Search, keep avoidence set($S$) when enumerating.
\end{remark}
\pagebreak
\subsection{Results}
\input{rhotauTable.tex}
\pagebreak

Previously, to find OTS, we enumerate all partitions of $\pa+\pb$. It cost much time. Discovering from tables in Results 3.3. We found the cycle property. Take $\pa=5$ as example.
\begin{tabular}{l|rrrrrrrrrrrrrrrrrrrr}
 $\tau(\pa,\pb)$ &  1 &  2 &  3 &  4 &  5 &  6 &  7 &  8 &  9 & 10 & 11 & 12 & 13 & 14 & 15 & 16 & 17 & 18 & 19 & 20 \\
\hline
 5        &  3 &  4 &  4 &  5 &  5 &  6 &  6 &  8 &  7 & 10 &  8 & 11 &  9 & 12 & 10 & 13 & 11 & 14 & 12 & 15 \\
\hline
\end{tabular}

From $\pb\geq 9$, $\tau(\pa,\pb)$ forms cycle of length $2$. In general, when $\pb$ is large enough $\tau(\pa,\pb)$ would forms into cycles.
Using this property to improve our algorithm for the case $\pb$ is much larger than $\pa$. The improved algorithm runs in time complexity with no $\pb$.

\begin{definition}
$d$ is the smallest non-factor of $\pa$
\end{definition}
\begin{lemma}
$\rho(a,b)$ has lower-bound $\ceil{\frac{b}{d}}$
\end{lemma} 

\begin{proof}
To proof a lower-bound, we show a case whose number of partitions=$\ceil{\frac{b}{d}}$
The scheme is trivial, divide $a+b$ into $\ceil{\frac{b}{d}}-1$ instances of $d$ and a $a+b-d\ceil{\frac{b}{d}}-1$ (denoted by $s$)

Then we shall prove the scheme avoids $\pa$ by contradiction.There are two cases.
If the subset contains $s$ but $s>a$.
If the subset doesn't contains $s$, it contains only several '$d$'s, but $d\not| a$.
Both cases raise contradiction.
\end{proof}
We shall setup notations before showing the folloing lemmas.  For a multiset $S$, 
We denote the number of $i$ in $S$ as $X_i$, that is, we have $X_i$ '$i$'(s) in $S$ and $X = \sum_i{X_i}$.
$S_k$ to denote the first $k$ elements(e.g. prefix) of $S$. $S_0$ is the empty set.

To avoid ambiguity with union operation, we use $S$\d{+} a number $q$ to denote the result multiset that adding $q$ to each element to multiset $S$.

We define set $U(S)$ for a multiset $s$ as subset-sums$\leq a$ of $S$. For example, $\pa=5$, $U({1,3,3})=\text{ignore}_{>5}(\{ 0,1,3,4,6,7\})=\{ 0,1,3,4\}$. Note that for any $S$, $U(S)$ always include zero.

\begin{lemma}
If a multiset $S$ avoids $a$ and composed only by factors of $a$, $|S|<a$
\end{lemma}

\begin{proof}

We shall prove this by contradiction

We divide the proof into two parts.

{\bf(First Part)} 
\begin{equation}\label{inceq}
\forall k, |U(S_{k+1})|>|U(S_k)|
\end{equation}

For simplisty we Denote $U(S_{k+1})$ as $u'$ and $U(S_k)$ as $u$, the k+1th element of $S$ as $w$.
Note that since $\text{ignore}_{>\pa}(u\text{\d{+}}w)\subseteq u'$
\begin{equation}\label{upro}
\forall i\geq w, i \in u\implies i+w\in u'
\end{equation}

We shall prove the statement \ref{inceq} by contradiction, if $|u'|=|u|$ for some $k$, then since $u\subseteq u'$, $u=u'$. And from statement \ref{upro}, we get
\[
\forall i\geq w, i \in u\implies i+w\in u'\implies i+w\in u
\]
Since U always include 0, and $0\in u \implies tw\in u$ for any $t$.
But $w$ is a factor of $\pa$, thus $\pa\in u$, contradicts $S$ avoids $\pa$

{\bf(Second Part)} We discover $U(S_0)$ to $U(S_{|S|})$, since $U(S_0)=\{0\}$, each time
$U$ at least a new element. If $|S|\geq \pa$, then $U(S)$
must be $\{0,1,...,\pa\}$, contradicts $S$ avoids $\pa$


\end{proof}

\begin{theorem} 
$\forall \pb>d(d+1)(\pa-1)  + (\pa - (\pa\mod d)$, an optimal partition contains at least $\floor{{\frac{\pa}{d}}}$ instances of $d$
\end{theorem}

\begin{proof} 
  We shall prove by contradiction.
  Let $S$ is a partition of $\pa+\pb$ avoids $\pa$.

  Assume $X_d<\lfloor{\frac{\pa}{d}} \rfloor $ then 
  \[
  X=\sum_{i<d}{X_i}+X_d+\sum_{i>d}{X_i}
  \]
  Since $\sum_i{iX_i}=\pa+\pb$, to maximize $X$, we use as much small number as possible. Hence

  \begin{align*}
  X&\leq \pa-1+ \floor{\frac{\pa}{d}}   +\floor{{\frac{\pb-(\pa - (\pa\mod d))}{d+1}}}\\
   &\leq \pa-1+ \frac{\pa}{d}   +{\frac{\pb-(\pa - (\pa\mod d))}{d+1}}
  \end{align*}
  
  $\forall \pb>d(d+1)(\pa-1) + (\pa - (\pa\mod d)), $
  \[X<\pa+\floor{\frac{\pa}{d}} + \pa d -d \leq \ceil{\frac{\pb}{d}}
  \]
  which is not optimal from lemma \ref{upper bound}.A contradiction occurs.
\end{proof}

\begin{theorem}
	$\forall \pb>d(d+1)(\pa-1)  + (\pa - (\pa\mod d)$, $\rho(\pa,\pb+d)=\rho(\pa,\pb)+1$
\end{theorem}

\begin{proof}
    $b>d(d+1)(a-1)  + (a - (a\mod d)$,
    let $S$ be a optimal partition of a+b avoids $a$, $S'=S+$'d' which is a partition of a+b+d.

    We divide the proof into avoidance part and optimal part

    {\bf(Avoidance)}
    prove by contradiction

    If $S'$ has subset-sum $\pa$, $S$ must has subset-sum $\pa$ or $\pa-d$. The former is not possible from definition. 
    If the later happens, $S$ has subset-sum $\pa$ since it has at least $\floor{{\frac{a}{d}}}$ instances of 'd', and $\pa-d$ couldn't has all of them. This contradicts definition.
    Thus, $S'$ avoids $\pa$.

    {\bf(Optimal)}
        prove by contradiction

        If $S'$ is not optimal there exist an optimal partition $P'$ that $|P'|>|S'|$ and has no subset-sum=$a$. $P'$ has at least one $d$, let $P=P'-d$(note $sum(P)=sum(S)$) then $P$ has no subset-sum=$a$ and $|P|=|P'|-1>|S'|-1=|S|$, which contradicts $S$ is optimal. Thus, $S'$ is optimal.
\\
    Thus the theorem holds.
\end{proof}

\begin{corollary}
for a constant $\pa$, $\rho(\pa,\pb)$ forms cycle of length $d$ if $\pb$ is sufficiently large
\end{corollary}

\subsection{Algorithm using cycle property}

\begin{algorithm}
\caption{Find optimal foolproof scheme using cycle property}\label{alg:cycle}
\begin{algorithmic}
  \State $K=\pa+\pb-kd>d(d+1)a$ \Comment{choose maximal $k$, e.g. minimal $K$}
  \Procedure{Find\_optimal\_foolproof\_scheme}{$\pa,\pb$}\Comment{return a OFS}
    \State{$R=$Find\_optimal\_foolproof\_scheme$^2$($\pa,\pb-Kd$)}\Comment{This procedure is from algorithm \ref{alg:naive}}
    \State{\textbf{return} $R$+k instances of 'd'}
  \EndProcedure
\end{algorithmic}
\end{algorithm}
check P's avoidence by Dynamic-Programming as solving discrete knapsack problem, keep avoidence vector when enumerating.

Time Complexity:

\[
O(k+a\times p_{aoivds\ \pa}(K))\\
= O(\pa\times p(K))\\
= O(\pa\frac{exp(\sqrt{\frac{2(1+d(d+1)(\pa-1)+\pa)}{3} } )}{1+d(d+1)(\pa-1)+\pa})
= O(\frac{exp(\sqrt{\frac{2(1+d(d+1)(\pa-1)+\pa)}{3} } )}{d^2})
\]
