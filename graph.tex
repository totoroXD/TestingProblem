\section{An Equivalence Graph Problem}


\subsection*{Idea}
We can present a set of comparisons as a graph. Vertice are the coins, edges are the comparisons.

Consider a graph ${\cal G}(\pa,\pb)$ contains two part, one is $K_\pa$, another is $K_\pb$, then we want to find a graph $g(\pa,\pb)$ (number of vertices $\leq \pa+\pb$), and $g(\pa,\pb)$ is not a subgraph of ${\cal G}(\pa,\pb)$. 
Let $t =$ number of edges of $g(\pa,\pb)$, then through these $t$ comparisons, at least one of them is unbalanced.

\begin{proof}

Prove by contradiction.

Denote the real coins as '+' and the fake ones as '-'. 

We compare edges of g(a,b). If we found all balance in these t comparison, we can assign a '+'s and b '-'s into g(a,b), then g(a,b) is the subgraph of ${\cal G}(\pa,\pb)$.
\end{proof}

Intuitively, if finding unbalance in a comparison, the game ends. But sometimes we can point a pair is unbalance after k comparisons(k is the answer) before the unbalance appears. The following shows $k_{min}=t_{min}-1$, which means the answer for Problem 4 is $t_{min}-1$

\begin{lemma}
$t_{min} = k_{min}+1 $
\end{lemma}

\begin{proof}
We divide the proof of the equation into two inequations

If we could point out a pair $e$ is unbalanced through the optimal graph $F$(with $k_{min}$ comparisons).
Consider a graph with these $k_{min}$ edges + $e$ which is not a subgraph of ${\cal G}(\pa,\pb)$
\[t_{min}\leq k_{min}+1\]

For a optimal graph $E$(with $t_{min}$ edges) which is not a subgraph of ${\cal G}(\pa,\pb)$. Let $E'$ = ($E$ remove an arbitrary edge $e$), then through $E'$

If $E'$ has unbalance comparison, we conclude the one is unbalanced, otherwise we could conclude $e$ is unbalanced.

Either case shows $E'$ is enough to end the game.
\[k_{min} \leq t_{min}-1\]

From the above inequations, the lemma holds.

\end{proof}

\subsection*{Goal}
We want to find a graph $g(\pa,\pb)$ that has minimum edges and $g(\pa,\pb)$ is not a subgraph of ${\cal G}(\pa,\pb)$. Through testing $g(\pa,\pb)-$an arbitrary edge $e$, we can point out a pair is unbalanced and it's optimal.

If comparing the edges of a graph results in at least one unbalanced edge in any condition.
We say this graph is a testing solution, feasible. Otherwise it's infeasible.

A testing solution(or a solution) is presented by graph.

\begin{definition}
An optimal testing solution(denoted by $OTS$) is a solution with minimum number of edges(comparisons).
\end{definition}

An $OTS$ can presented by an integer partition as well, which is covered in Section 3.

\begin{definition}
$\tau(\pa,\pb)$ is the number of edges of $OTS(\pa,\pb)$
\end{definition}

For example, $\tau(1,2)=2$, since testing a pair may result in balanced. On the other hand, testing two non-duplicate pair, one of them must results in unbalanced.

In this case, testing a pair is enough to point out a real coin and a fake one. 

Denote the coins we test as $coin_a$ $coin_b$ , the coin we didn't test as $coin_c$
If $coin_a$ and $coin_b$ is not balanced, then point out $coin_a$ and $coin_b$. Otherwise, point out $coin_a$ and $coin_c$