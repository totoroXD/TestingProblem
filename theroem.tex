\begin{theorem}
Any graph with $n \leq \pa + \pb$ vertices and $m \leq \lfloor (\pa+\pb)/2 \rfloor-1$ edges $(0<\pa \leq \pb)$ 
is always a subgraph of ${\cal G}(\pa,\pb)$ (or equivalently, a subgraph of ${\cal G}(\pb,\pa)$).
\end{theorem}

\begin{proof}
We shall prove this theorem by induction.  \\

\noindent
{\bf (Basis Case:)} If $\pa = \pb =1$, then $\lfloor (\pa+\pb)/2 \rfloor-1 = 0$. Any graph with $n \leq 2$ vertices and $m \leq 0$ edges (i.e., no edges) is always a subgraph of ${\cal G}(1,1)$.
\\

\noindent
{\bf (Inductive Case:)} Suppose that the theorem holds for all $\pa + \pb \leq k$.  Our target is to show that the theorem also holds for the case $\pa + \pb = k + 1$ with $\pa \leq \pb$.   Consider a graph $G$ with  $n \leq k+1$ vertices and with $m \leq \lfloor (k+1)/2 \rfloor -1$ edges. 

\begin{enumerate}
  \item If $G$ is connected, then $n \leq m + 1 \leq (k+1)/2 \leq b$, which is a subgraph of $K_b$, 
           and thus a subgraph of ${\cal G}(\pa,\pb)$.   

  \item Otherwise, $G$ is not connected.  If $G$ has no edges, then $G$ is obviously a subgraph of ${\cal G}(\pa,\pb,0)$ 
          since $G$ has at most $\pa+\pb$ vertices.
          Else, let $C$ be the connected component of $G$ with the largest number $n'$ of vertices (so that $n' \geq 2$.
          Then, the number of edges in $C$ is at least $n'-1$.  To complete the proof, it is sufficient to show 
          that $G - C$ is a subgraph of ${\cal G}(\pa, \pb - n')$, as we can map $C$ as a subgraph in $K_b$.

The number of vertices in $G-C$ is $k+1-n' = \pa + (\pb-n')$, and the number of edges of $G-C$ is at most 
\begin{eqnarray*}
m - (n'-1) \leq \floor{(k+1)/2} - n' &=& \floor{(k+1)/2 - n'} \\
&=& \floor{ (k+1-n')/2 - n'/2 } \\
&\leq& \floor{ (k+1-n')/2 - 1 } \\
&=& \floor{(k+1-n')/2} - 1 = \floor{ (\pa + (\pb+n'))/2 } - 1.
\end{eqnarray*}
By induction hypothesis, $G-C$ is a subgraph of ${\cal G}(\pa, \pb-n')$, 
and consequently $G$ is a subgraph of ${\cal G}(\pa,\pb)$.
\end{enumerate}
In all cases, $G$ is a subgraph of ${\cal G}(\pa,\pb)$.  This completes the proof of the induction case, so that by the principle of mathematical induction, the theorem follows.
\end{proof}

\begin{corollary}
for all odd number $\pa,\pb$, $\tau(a,b)=\frac{\pa+\pb}{2}$
\end{corollary}

Consider a graph  $g(a,b)$ that have  $\frac{a+b}{2}$ components, each components have only two nodes. then  $g(\pa,\pb)$ is not a subgraph of  ${\cal G}(\pa,\pb)$, and by Thm2.2, it is the optimal solution.

\begin{theorem} 
For every $a,b$, we have a 2-approximation solution of  $\tau(\pa,\pb)$.
\end{theorem}
Consider a graph  $g(\pa,\pb)$ is a line with $max(\pa,\pb)$ nodes.  $g(\pa,\pb)$ is not a subgraph of  ${\cal G}(\pa,\pb)$, and by Thm2.2, it is a 2-approximation solution.
